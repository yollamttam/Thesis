\documentclass[11pt]{article}
\usepackage{matt}
\begin{document}

\section*{Working Thesis Outline}

\begin{enumerate}
\item [{Acknowledgements:}] I clearly needed a lot of help to get this done.
\item [{\bf Part I:}] Introduction and Context
\begin{enumerate}[1.]
\item[]
\item What is reionization? Why do we care?
\begin{enumerate}[A.]
\item CMB tells us that Universe is neutral at $z \sim 1080$
\item [$\to$] \textcolor{cyan}{How do we know this? Easy to explain?}
\item Nearby quasar spectra tell us hydrogen is ionized today
\item How did this happen? Interesting in its own right
\item Timing and nature of reionization affects structure formation as well
\end{enumerate}


%%% SUMMARY OF PROBES %%%
\item Selected summary of EoR probes
\begin{enumerate}[A.]
% Lya Forest %
\item \lya\ Forest
\begin{enumerate}[$\to$]
\item Ideal interpretations, complications with large $\tau$
\item Gunn-Peterson trough
\item Damping wing redward of \lya
\item Dark pixel covering fraction
\end{enumerate}
% CMB Constraints %
\item CMB Constraints
\begin{enumerate}[$\to$]
\item $\tau_{e}$
\item kSZ
\end{enumerate}
% Lya Emitters %
\item \lya\ Emitters (Lidz, Taylor)
\begin{enumerate}[$\to$]
\item \lya\ fraction
\item \lya\ clustering
\end{enumerate}
% 21-cm Line
\item 21-cm Line
\begin{enumerate}[$\to$]
\item 21 cm intensity mapping
\item Power Spectrum (Lidz, Aguirre, Moore)
\item Global 21 cm
\end{enumerate}
% Photon counting? %
\item Luminosity Function Measurements?
\begin{enumerate}[$\to$]
\item Photon counting to constrain reionization
\item GRBs
\end{enumerate}
\end{enumerate}
\item Summary of Constraints
\begin{enumerate}[A.]
\item Timing
\begin{enumerate}[$\to$]
\item A good plot to show might be Figure 3 of Robertson et al. (2015)
\end{enumerate}
\item Sources
\begin{enumerate}[$\to$]
\item Could describe arguments against quasars (due to insufficient abundance) 
\item and X-ray sources due to lack of presence in X-ray background
\item Robertson et al. (2013/2015) support galaxies primarily sourcing EoR
\item Could mention temperature/heating scenarios ruled out by PAPER
\end{enumerate}
\end{enumerate}
\end{enumerate}




%%%%%% PART II %%%%%%
\item [{\bf Part II:}] The \lya\ Forest is Lovely, Dark, and Deep...
\begin{enumerate}[A.]
\item Stacking spectra to constrain $\axhi$
\item Wavelet analysis to measure IGM temperature
\end{enumerate}
\item [{\bf Part III:}] The 21-cm Line
\begin{enumerate}[A.]
\item Blindly Identifying Ionized Regions in Noisy Redshifted 21cm Observations
\end{enumerate}
\item [{\bf Part IV:}] Conclusions
\begin{enumerate}[$\to$]
\item Wait, um... What goes in here?
\item Briefly summarize results of work contained 
\item Describe future direction, concerns that need to be addressed
\item Cite lots of stuff.
\end{enumerate}
\end{enumerate}






\section*{Papers to Review}
\begin{enumerate}
\item [{\bf Part I:}]
\begin{enumerate}[-]
\item 
\item Furlanetto, Oh, and Briggs review
\item Robertson et al. 2013/2015
\item Loeb and Furlanetto Books
\end{enumerate}
\item [{\bf Part II:}] 
\end{enumerate}

\section*{Questions to Answer for Yourself}
\begin{enumerate}[-]
\item What is the ``gravitational instability" model in the context of the \lya\ forest?
\end{enumerate}

\end{document}