
% Thesis Abstract -----------------------------------------------------

\begin{abstracts}

Based on observations of the early Universe, we know that shortly after the Big Bang, the Universe was composed almost entirely of neutral hydrogen and neutral helium. However, observations of nearby quasars suggest that the gas between galaxies today is neutral to less than one part in $10^{4}$.  Thus, it must be the case that some energetic process occurred that stripped the electrons from almost all atoms in the intergalactic medium. Understanding the timing and nature of this process, dubbed ``reionization'', is one of the great outstanding problems in astrophysics and cosmology today. In this thesis, we develop several methods for utilizing existing and future measurements in order to make progress toward this end.


We begin by proposing two novel approaches for searching for signatures of underlying neutral hydrogen in the \lya\ and \lyb\ forest of distant quasars. We show that, if the Universe is $\gtrsim 5\%$ neutral at $z \sim 5.5$, then damping-wing absorption from neutral hydrogen and absorption from primordial deuterium should leave an observable imprint in the \lya\ and \lyb\ forest, respectively. Furthermore, the presence of neutral islands should qualitatively alter the size distribution of absorbed regions.


We continue by discussing the ability for the intergalactic medium to retain a thermal memory of the reionization process at redshifts $z \sim 5$ which in turn affects the small-scale structure in the \lya\ forest. Motivated by this, we model the temperature of the intergalactic medium after reionization and develop a temperature measurement technique that should be able to distinguish between  scenarios where reionization ends at $z \sim 6$ and at $z \sim 10$. 


Lastly, we turn our attention to 21-cm observations during reionization. We demonstrate that, while precise mapping of 21-cm emission from neutral hydrogen should be infeasible by first and second generation interferometers, it may be possible to make \textit{crude} maps of the reionization process and identify individual ionized regions. This would provide us with direct confirmation that we are observing reionization and provide information as to its timing and the nature of the ionizing sources.




\end{abstracts}

% ---------------------------------------------------------------------- 
